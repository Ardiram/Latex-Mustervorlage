\begin{table}[H]
    \centering
    \caption{Die gemessenen Winkel und Abstände der optimierten Geometrien. Die farbigen Atome zeigen die zwischen denen die jeweilige Größe gemessenen wurde. Die blauen Atome sind von dem 2,5-Dioxo-2,5-dihydrofuran-3-carbonitril und die grünen sind von dem 2,3-Dichlorcyclopenta-1,3-dien.}
    \label{tab:geometrie}
    \resizebox{\textwidth}{!}{%
    \begin{tabular}{@{}l S[table-format=3.3] S[table-format=3.3] S[table-format=1.3] S[table-format=1.3]@{}}
    \toprule
                     & {Diederwinkel / \si{\degree}}        & {Winkel / \si{\degree}}     & \multicolumn{2}{c}{intramolekulare Distanz / \si{\angstrom}} \\ \midrule
                     & {\ch{O=}{\color[HTML]{0099FF}C}\ch{-}{\color[HTML]{0099FF}C}(CN)\ch{-}{\color[HTML]{0099FF}C}H\ch{-}{\color[HTML]{0099FF}C}\ch{=O}} 
                     & {H{\color[HTML]{00CC66}C}\ch{-}{\color[HTML]{00CC66}C}H\textsubscript{2}\ch{-}{\color[HTML]{00CC66}C}H} 
                     & {NC\ch{-}{\color[HTML]{0099FF}C}\ch{-}{\color[HTML]{00CC66}C}H}        
                     & {H{\color[HTML]{0099FF}C}\ch{-}{\color[HTML]{00CC66}C}H}               \\
    Präkomplex       & 179.635                            & 103.941                   & 3.171                          & 3.159                         \\
    Übergangszustand & 175.785                            & 99.973                    & 2.309                          & 2.038                         \\
    Produkt          & 177.903                            & 93.887                    & 1.590                          & 1.570                         \\ 
    \bottomrule
    \end{tabular}%
    }
\end{table}