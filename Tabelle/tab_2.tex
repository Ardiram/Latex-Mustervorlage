\begin{table}[H]
    \caption{Die berechneten Energiedifferenzen in \unit{\kilo\joule\per\mole}.}
    \centering
    \begin{threeparttable}
    \label{tab:energie}
    \begin{tabular}{@{}l *{4}{S[table-format=-3.2]} @{}}
    \toprule
                                  & {DFT}         & {DFT + ZPE}    & {CC}          & {CC + ZPE}     \\
                                  & \multicolumn{4}{|c|}{\unit{\kilo\joule\per\mol}} \\ \midrule
    \makecell{Edukte - Präkomplex \\ (\textbf{Assoziationsenergie})}          &  -37.24        &  -34.02         &  -32.39        &  -29.17         \\
    \makecell{Edukte - Übergangszustand \\ (\textbf{Aktivierungsenergie})} & 16.40        & 23.07         & 20.78        & 27.45         \\
    \makecell{Edukte - Produkt \\ (\textbf{thermodynamische Energie}\textsuperscript{a})}             &  -90.75        &  -72.57         & -135.71        & -117.52         \\
    \bottomrule
    \end{tabular}
    
    \begin{tablenotes}
    \item[a] Auch als Reaktionsenthalpie \textDelta H bekannt, wird aber im weiteren Verlauf mit \textDelta E bezeichnet
    \end{tablenotes}
    \end{threeparttable}
\end{table}